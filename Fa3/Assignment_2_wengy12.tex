\documentclass[11pt,fleqn]{article}

\setlength {\topmargin} {-.15in}
\setlength {\textheight} {8.6in}

\usepackage{amsmath}
\usepackage{amssymb}
\usepackage{amsthm}
\usepackage{color}

\renewcommand{\labelenumi}{\theenumi.}
\renewcommand{\labelenumii}{\theenumii.}
\renewcommand{\labelenumiii}{\theenumiii.}
\newcommand{\be}{\begin{enumerate}}
\newcommand{\ee}{\end{enumerate}}
\newcommand{\bi}{\begin{itemize}}
\newcommand{\ei}{\end{itemize}}
\newcommand{\bc}{\begin{center}}
\newcommand{\ec}{\end{center}}
\newcommand{\bsp}{\begin{sloppypar}}
\newcommand{\esp}{\end{sloppypar}}
\newcommand{\mname}[1]{\mbox{\sf #1}}
\newcommand{\pnote}[1]{{\langle \text{#1} \rangle}}

\begin{document}

\begin{center}

  {\large \textbf{COMPSCI/SFWRENG 2FA3}}\\[2mm]
  {\large \textbf{Discrete Mathematics with Applications II}}\\[2mm]
  {\large \textbf{Winter 2019}}\\[8mm]
  {\huge \textbf{Assignment 2}}\\[6mm]
  {\large \textbf{Dr.~William M. Farmer}}\\[2mm]
  {\large \textbf{McMaster University}}\\[6mm]
  {\large Revised: January 24, 2019}

\end{center}

\medskip

Assignment 2 consists of two problems.  You must write your solutions
to the problems using LaTeX.

Please submit Assignment~2 as two files,
\texttt{Assignment\_2\_\emph{YourMacID}.tex} and
\texttt{Assignment\_2\_\emph{YourMacID}.pdf}, to the Assignment~2
folder on Avenue under Assessments/Assignments.
\texttt{\emph{YourMacID}} must be your personal MacID (written without
capitalization).  The \texttt{Assignment\_2\_\emph{YourMacID}.tex}
file is a copy of the LaTeX source file for this assignment
(\texttt{Assignment\_2.tex} found on Avenue under
Contents/Assignments) with your solution entered after each problem.
The \texttt{Assignment\_2\_\emph{YourMacID}.pdf} is the PDF output
produced by executing

\begin{itemize}

  \item[] \texttt{pdflatex Assignment\_2\_\emph{YourMacID}}

\end{itemize}

This assignment is due \textbf{Sunday, February 3, 2019 before
  midnight.}  You are allow to submit the assignment multiple times,
but only the last submission will be marked.  \textbf{Late submissions
  and files that are not named exactly as specified above will not be
  accepted!}  It is suggested that you submit your preliminary
\texttt{Assignment\_2\_\emph{YourMacID}.tex} and
\texttt{Assignment\_2\_\emph{YourMacID}.pdf} files well before the
deadline so that your mark is not zero if, e.g., your computer fails
at 11:50 PM on February 3.

\textbf{Although you are allowed to receive help from the
  instructional staff and other students, your submission must be your
  own work.  Copying will be treated as academic dishonesty! If any of
  the ideas used in your submission were obtained from other students
  or sources outside of the lectures and tutorials, you must
  acknowledge where or from whom these ideas were obtained.}

\newpage

\subsection*{Background}

Let $(S,<)$ be a strict partial order.  $(S,<)$ is \emph{dense} if,
for all $x,y \in S$ with $x < y$, there is some $z \in S$ such that $x
< z < y$. The strict total order $(\mathbb{Q}, <_{\rm rat})$ of the
rationals and the strict total order $(\mathbb{R},<_{\rm real})$ of
the real numbers are both dense.

\subsection*{Problems}

\be

  \item \textbf{[10 points]} Let \mname{SimpleTree} be the inductive
    set defined by the following constructors:

  \be

    \item $\mname{Leaf} : \mathbb{N} \rightarrow \mname{SimpleTree}$.

    \item $\mname{Branch1} : \mname{SimpleTree} \rightarrow
      \mname{SimpleTree}$.

    \item $\mname{Branch2} : \mname{SimpleTree} \times
      \mname{SimpleTree} \rightarrow \mname{SimpleTree}$.

  \ee

  The function $\mname{leafs} : \mname{SimpleTree} \rightarrow
    \mathbb{N}$ is defined by pattern matching as:

  \be

    \item $\mname{leafs}(\mname{Leaf}(n)) = 1$.

    \item $\mname{leafs}(\mname{Branch1}(t)) = \mname{leafs}(t)$.

    \item $\mname{leafs}(\mname{Branch2}(t_1,t_2)) =
      \mname{leafs}(t_1) + \mname{leafs}(t_2)$.

  \ee

  The function $\mname{branches} : \mname{SimpleTree} \rightarrow
    \mathbb{N}$ is defined by pattern matching as:

  \be

    \item $\mname{branches}(\mname{Leaf}(n)) = 0$.

    \item $\mname{branches}(\mname{Branch1}(t)) = 1 + \mname{branches}(t)$.

    \item $\mname{branches}(\mname{Branch2}(t_1,t_2)) = 1 +
      \mname{branches}(t_1) + \mname{branches}(t_2)$.

  \ee

  Prove that, for all $t \in \mname{SimpleTree}$, \[\mname{leafs}(t) \le
  \mname{branches}(t) + 1.\]

  \textcolor{blue}{\textbf{Yunbing Weng, 400158853, 03/02/2019}}

\begin{proof}
Let $P(t) \equiv \mname{leafs}(t) \le \mname{branches}(t) + 1$. We will proof P(t) for all $t \in \mname{SimpleTree}$ using Structural induction.

\medskip

\emph{Base case:} Prove $P(\mname{Leaf(n)})$
\begin{align*}
P(\mname{Leaf(n)}) \equiv \,& \mname{leafs(\mname{Leaf(n)})} \le \mname{branches(\mname{Leaf(n)})} + 1 &\pnote{definition of P}\\
\equiv \, &1 \le 0 + 1 &\pnote{definition of \mname{leafs} and \mname{branches} }\\
\equiv \, &1 \le 1 &\pnote{arithmetic}\\
\end{align*}

\emph{Induction step:}\\
\emph{Case 1:} $t = \mname{Branch1(t1)}$. We assume t1 and prove $P(\mname{Branch1(t1)})$.
\begin{align*}
&P(\mname{\mname{Branch1(t1)}})\\
&\equiv \, \mname{leafs(\mname{Branch1(t1)})} \le \mname{branches(\mname{Branch1(t1)})} + 1 &\pnote{definition of P}\\
&\equiv \, \mname{leafs(t1)} \le 1+ \mname{branches(t1)} + 1 &\pnote{definition of \mname{leafs} and \mname{branches}}\\
&\equiv \, \mname{branches(t1)} + 1 \le 1+ \mname{branches(t1)} + 1 &\pnote{induction hypothesis}\\
&\equiv \, \mname{branches(t1)} + 1 \le \mname{branches(t1)} + 2 &\pnote{arithmetic}\\
\end{align*}

\emph{Case 2:} $t = \mname{Branch2(t1, t2)}$. We assume t1 and t2, and prove $P(\mname{Branch2(t1, t2)})$\\
\begin{flalign*}
&P(\mname{\mname{Branch2(t1, t2)}})\\
&\equiv \mname{leafs(\mname{Branch2(t1, t2)})} \le \mname{branches(\mname{Branch2(t1, t2)})} + 1 &\pnote{definition of P}\\
&\equiv \, \mname{leafs(t1)} + \mname{leafs(t2)} \le \mname{branches(t1)} + \mname{branches(t2)} + 2 &\pnote{definition of \mname{leafs} and  and \mname{branches}}\\
&\equiv \, \mname{branches(t1)} + \mname{branches(t2)} + 2 \le 2 + \mname{branches(t1)} + \mname{branches(t2)} &\pnote{induction hypothesis}\\
\end{flalign*}

\end{proof}
  \bigskip

  \item \textbf{[10 points]} Let $(S,<)$ be a strict total order such
    that there exist $a,b \in S$ with $a < b$ (i.e., $S$ has at least
    two members).  Show that, if $(S,<)$ is dense, then $(S,<)$ is not
    a well-order.

  \bigskip

  \textcolor{blue}{\textbf{Yunbing Weng, 400158853, 03/02/2019}}

\begin{proof}

  $(S,<)$ is dence strict total order with at least two members.
  To show that it is not well ordered, we just need to find an infinate descending sequence.
  
  First element will be \mname{x}(0) = $b$, then, second element will be between $a$ and $b$, because it is strict dense order,
between any two elements must exist one more. According to the task, we have at least a and b, so we can find \mname{x}(1) between them.
Also it should be obvious that $a$  $<$ \mname{x}(1) $< b$ = \mname{x}(0).

  \mname{x}(i) will be between a and \mname{x}(i-1).

\mname{x}(i) exists because of the way we built it, our first two elements exist, therefore each next will
exist ($(S,<)$ is dence). It is also not hard to see that $a <$ \mname{x}(i) $<$ \mname{x}(i-1) 

We can notice that \mname{x}(i) is infinate sequance (because it is dense). Also it doent have
least element because each next element is less then previous. 

Therefore, $(S,<)$ is not well-ordered

\end{proof}

\ee

\end{document}


