\documentclass[11pt,fleqn]{article}

\setlength {\topmargin} {-.15in}
\setlength {\textheight} {8.6in}

\usepackage{amsmath}
\usepackage{amssymb}
\usepackage{color}
\usepackage{tikz}
\usepackage{amsthm}
\usetikzlibrary{automata,positioning,arrows}
\usepackage{diagbox}
\usepackage{stackrel}

\renewcommand{\labelenumi}{\theenumi.}
\renewcommand{\labelenumii}{\theenumii.}
\renewcommand{\labelenumiii}{\theenumiii.}
\newcommand{\be}{\begin{enumerate}}
\newcommand{\ee}{\end{enumerate}}
\newcommand{\bi}{\begin{itemize}}
\newcommand{\ei}{\end{itemize}}
\newcommand{\bc}{\begin{center}}
\newcommand{\ec}{\end{center}}
\newcommand{\bsp}{\begin{sloppypar}}
\newcommand{\esp}{\end{sloppypar}}
\newcommand{\mname}[1]{\mbox{\sf #1}}
\newcommand{\sB}{\mbox{$\cal B$}}
\newcommand{\sC}{\mbox{$\cal C$}}
\newcommand{\sF}{\mbox{$\cal F$}}
\newcommand{\sM}{\mbox{$\cal M$}}
\newcommand{\sP}{\mbox{$\cal P$}}
\newcommand{\sV}{\mbox{$\cal V$}}
\newcommand{\set}[1]{{\{ #1 \}}}
\newcommand{\Neg}{\neg} 
\ifdefined \And 
\renewcommand{\And}{\wedge}
\else
\newcommand{\And}{\wedge}
\fi
\newcommand{\Or}{\vee}
\newcommand{\Implies}{\Rightarrow}
\newcommand{\Iff}{\Leftrightarrow}
\newcommand{\Forall}{\forall}
\newcommand{\ForallApp}{\forall\,}
\newcommand{\Forsome}{\exists}
\newcommand{\ForsomeApp}{\exists\,}
\newcommand{\mdot}{\mathrel.}
\newcommand{\der}[2]{\stackrel[#2]{#1}{\longrightarrow}}

\begin{document}

\begin{center}

  {\large \textbf{COMPSCI/SFWRENG 2FA3}}\\[2mm]
  {\large \textbf{Discrete Mathematics with Applications II}}\\[2mm]
  {\large \textbf{Winter 2019}}\\[8mm]
  {\huge \textbf{Assignment 9}}\\[6mm]
  {\large \textbf{Dr.~William M. Farmer}}\\[2mm]
  {\large \textbf{McMaster University}}\\[6mm]
  {\large Revised: March 22, 2019}

\end{center}

\medskip

Assignment 9 consists of two problems.  You must write your solutions
to the problems using LaTeX.

Please submit Assignment~9 as two files,
\texttt{Assignment\_9\_\emph{YourMacID}.tex} and
\texttt{Assignment\_9\_\emph{YourMacID}.pdf}, to the Assignment~9
folder on Avenue under Assessments/Assignments.
\texttt{\emph{YourMacID}} must be your personal MacID (written without
capitalization).  The \texttt{Assignment\_9\_\emph{YourMacID}.tex}
file is a copy of the LaTeX source file for this assignment
(\texttt{Assignment\_9.tex} found on Avenue under
Contents/Assignments) with your solution entered after each problem.
The \texttt{Assignment\_9\_\emph{YourMacID}.pdf} is the PDF output
produced by executing

\begin{itemize}

  \item[] \texttt{pdflatex Assignment\_9\_\emph{YourMacID}}

\end{itemize}

This assignment is due \textbf{Sunday, March 31, 2019 before
  midnight.}  You are allow to submit the assignment multiple times,
but only the last submission will be marked.  \textbf{Late submissions
  and files that are not named exactly as specified above will not be
  accepted!}  It is suggested that you submit your preliminary
\texttt{Assignment\_9\_\emph{YourMacID}.tex} and
\texttt{Assignment\_9\_\emph{YourMacID}.pdf} files well before the
deadline so that your mark is not zero if, e.g., your computer fails
at 11:50 PM on March 31.

\textbf{Although you are allowed to receive help from the
  instructional staff and other students, your submission must be your
  own work.  Copying will be treated as academic dishonesty! If any of
  the ideas used in your submission were obtained from other students
  or sources outside of the lectures and tutorials, you must
  acknowledge where or from whom these ideas were obtained.}

\newpage

\subsection*{Problems}

\be

  \item \textbf{[10 points]} Let $G = (N,\Sigma,P,S)$ be the CFG
    where $N = \set{S}$, $\Sigma = \set{a,b}$, and $P$ contains the
    following productions:

  \begin{itemize}

    \item[] $S \rightarrow aSb \mid \epsilon$.

  \end{itemize}

  For $x \in \Sigma^*$, let $P(x)$ be the property that $S \der{*}{G}
  x$ iff $x = a^nb^n$ for some $n \ge 0$.  Prove $\ForallApp x \in
  \Sigma^* \mdot P(x)$ by weak induction on the length of the
  derivation $S \der{*}{G} x$ for the ($\Rightarrow$) direction and by
  strong induction on the length of $x$ for the ($\Leftarrow$) direction.

  \bigskip

  \textcolor{blue}{\textbf{Yunbing Weng, wengy12, 03/31/2019}}

\begin{proof}
  	
	$S \der{n}{G} x \Rightarrow x = a^nb^n$ for all $n \in \mathbb{N}$ by weak induction.\\

  	\medskip
  	
  	\emph{Base case}: $n = 0$.

	$S \der{0}{G}  \epsilon \Rightarrow x = \epsilon$

	\medskip

	\emph{Induction step}: $n \ge 0$. Assume $P(n)$. We must show $P(n + 1)$.


	$S \der{n+1}{G} x_{n+1} \Leftarrow S \der{1}{G} aSb \land aSb \der{n}{G} x_{n+1}$ (definition)

	$= S \der{1}{G} aSb \land aSb \der{n}{G} ax_{n}b$ (Induction hypothesis)

	$= S \der{1}{G} aSb \land aSb \der{n}{G} aa^nb^nb$ (Induction hypothesis)

	$= S \der{1}{G} aSb \land aSb \der{n}{G} a^{n+1}b^{n+1}$ (Arithmetic)

	$= S \der{n+1}{G}  a^{n+1}b^{n+1}$  (definition)

\end{proof}

\begin{proof}

	$S \der{*}{G} x \Leftarrow x = a^nb^n$ for all $n \in \mathbb{N}$ by strong induction.\\
  
	\medskip

	\emph{Base case}: $n = 0$.

	x = $a^0b^0$ = $\epsilon$ = $S \der{0}{G}  \epsilon$

	\emph{Induction step}: $n \ge 0$. Assume $\forall i \in  \mathbb{N}| i \le n.P(n)$. We must show $P(n + 1)$.

	x = $a^{n+1}b^{n+1}$

	= $aa^nb^nb$ (Arithmetic)

	= $aSb$ (Induction hypothesis)
 
 	$\leftarrow$ S (Definition)


\end{proof}


  \item \textbf{[10 points]} Let $\Sigma = \set{a,b}$ and $L = \set{x
    \in \Sigma^* \mid x \not= \mname{rev}(x)}$.  Construct a simple
    grammar for $L$ as well as grammars in Chomsky and Greibach normal
    form for $L$.

  \bigskip

  \textcolor{blue}{\textbf{Yunbing Weng, wengy12, 03/31/2019}}

  simple  grammar:
	
	 Let G = (N,$\set{a,b}$,P,S)

  N = {S, R, T}

  P = 

	$S \rightarrow aSa | bSb | T$
 
	$T \rightarrow aRb | bRa$

	$R \rightarrow aRa | bRb | T | a | b | \epsilon$

Chomsky:

N = {$S, S_a, S_b, T, R, R_a, R_b, A, B, E$}

P = 

	$S \rightarrow AS_a|AS_b|TE$

	$S_a\rightarrow SA$

	$S_b \rightarrow SB$

	$T \rightarrow AR_b|BR_a$

	$R_a \rightarrow RA$

	$R_b \rightarrow RB$

	$R \rightarrow AR_a | BR_b | TE | a | b | \epsilon$

	$A \rightarrow a$

	$B \rightarrow b$

	$E \rightarrow \epsilon$

Greibach:

N = {S, T, R, A, B}

P = 

	$S \rightarrow aSA|aSB|aTB|bTA$

	$T \rightarrow aRB|bRA$

	$R \rightarrow aRA|bRB|bRA|aRB|a|b|\epsilon$

	$A \rightarrow a$

	$B \rightarrow b$
  

\ee

\end{document}
































