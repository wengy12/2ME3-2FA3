\documentclass[11pt,fleqn]{article}

\setlength {\topmargin} {-.15in}
\setlength {\textheight} {8.6in}

\usepackage{amsmath}
\usepackage{amssymb}
\usepackage{amsthm}
\usepackage{color}

\renewcommand{\labelenumi}{\theenumi.}
\renewcommand{\labelenumii}{\theenumii.}
\renewcommand{\labelenumiii}{\theenumiii.}
\newcommand{\be}{\begin{enumerate}}
\newcommand{\ee}{\end{enumerate}}
\newcommand{\bi}{\begin{itemize}}
\newcommand{\ei}{\end{itemize}}
\newcommand{\bc}{\begin{center}}
\newcommand{\ec}{\end{center}}
\newcommand{\bsp}{\begin{sloppypar}}
\newcommand{\esp}{\end{sloppypar}}
\newcommand{\mname}[1]{\mbox{\sf #1}}
\newcommand{\pnote}[1]{{\langle \text{#1} \rangle}}
\newcommand{\sB}{\mbox{$\cal B$}}
\newcommand{\sC}{\mbox{$\cal C$}}
\newcommand{\sF}{\mbox{$\cal F$}}
\newcommand{\sP}{\mbox{$\cal P$}}
\ifdefined \And 
\renewcommand{\And}{\wedge}
\else
\newcommand{\And}{\wedge}
\fi

\begin{document}

\begin{center}

  {\large \textbf{COMPSCI/SFWRENG 2FA3}}\\[2mm]
  {\large \textbf{Discrete Mathematics with Applications II}}\\[2mm]
  {\large \textbf{Winter 2019}}\\[8mm]
  {\huge \textbf{Assignment 3}}\\[6mm]
  {\large \textbf{Dr.~William M. Farmer}}\\[2mm]
  {\large \textbf{McMaster University}}\\[6mm]
  {\large Revised: February 1, 2019}

\end{center}

\medskip

Assignment 3 consists of four problems.  You must write your solutions
to the problems using LaTeX.

Please submit Assignment~3 as two files,
\texttt{Assignment\_3\_\emph{YourMacID}.tex} and
\texttt{Assignment\_3\_\emph{YourMacID}.pdf}, to the Assignment~3
folder on Avenue under Assessments/Assignments.
\texttt{\emph{YourMacID}} must be your personal MacID (written without
capitalization).  The \texttt{Assignment\_3\_\emph{YourMacID}.tex}
file is a copy of the LaTeX source file for this assignment
(\texttt{Assignment\_3.tex} found on Avenue under
Contents/Assignments) with your solution entered after each problem.
The \texttt{Assignment\_3\_\emph{YourMacID}.pdf} is the PDF output
produced by executing

\begin{itemize}

  \item[] \texttt{pdflatex Assignment\_3\_\emph{YourMacID}}

\end{itemize}

This assignment is due \textbf{Sunday, February 10, 2019 before
  midnight.}  You are allow to submit the assignment multiple times,
but only the last submission will be marked.  \textbf{Late submissions
  and files that are not named exactly as specified above will not be
  accepted!}  It is suggested that you submit your preliminary
\texttt{Assignment\_3\_\emph{YourMacID}.tex} and
\texttt{Assignment\_3\_\emph{YourMacID}.pdf} files well before the
deadline so that your mark is not zero if, e.g., your computer fails
at 11:50 PM on February 10.

\textbf{Although you are allowed to receive help from the
  instructional staff and other students, your submission must be your
  own work.  Copying will be treated as academic dishonesty! If any of
  the ideas used in your submission were obtained from other students
  or sources outside of the lectures and tutorials, you must
  acknowledge where or from whom these ideas were obtained.}

\newpage

\subsection*{Background}

Let $\Sigma = (\sB,\sC,\sF,\sP,\tau)$ be a finite signature of MSFOL,
$F_{\Sigma}$ be the set of $\Sigma$-formulas, and $A \in F_{\Sigma}$.
Recall that the members of $F_{\Sigma}$ are certain strings of
symbols.  A \emph{subformula} of $A$ is a $B \in F_{\Sigma}$ such that
$B$ is a substring of $A$.  For example, let $A$ be the formula $((0 =
2) \And (3 \mid 4))$.  Then ``$(0 = 2)$'', ``$(3 \mid 4)$'', and ``$((0 = 2)
\And (3 \mid 4))$'' are the subformulas of $A$, and ``$(0 = {}$'' and
``$\And$'' are two substrings of $A$ that are not subformulas of $A$.

\subsection*{Problems}

\be

  \item \textbf{[5 points]} Let $\mname{subformulas} : F_{\Sigma}
    \rightarrow \mname{Set}(F_{\Sigma})$ be the function that maps a
    formula $A \in F_{\Sigma}$ to the set of subformulas of $A$.
    Define $\mname{subformulas}$ by recursion using pattern matching.


  \textcolor{blue}{\textbf{Yunbing Weng, wengy12, 2/10/2019}}


 \begin{enumerate}
    \item  $\mname{subformulas}(t_1 = t_2) = (t_1 = t_2)$ 
    \item  $\mname{subformulas}(p(t_1, ..., t_n)) = p(t_1, ..., t_n)$
    \item  $\mname{subformulas}(\lnot A) = \lnot A$ $\cup$ $\mname{subformulas}(A)$
    \item  $\mname{subformulas}(Q^*x: \alpha .A)$ = $(Q^*x: \alpha .A)$ $\cup$ $\mname{subformulas}(A) $
    \item  $\mname{subformulas}(A \S^{**} B)$ = $A \S^{**} B$ $\cup$ $\mname{subformulas}(A)$ $\cup$ $\mname{subformulas}(B)$  
\end{enumerate}

Where

*Q - any quatifier, like $\forall, \exists, etc.$

**$\S$ - any logical constant, like $\lor, \land, \implies, etc.$
    
  

  \item \textbf{[5 points]} For $A \in F_{\Sigma}$, let $a(A)$ and
    $b(A)$ be the number of equalities and predicate applications
    occurring in $A$, respectively.  Prove that, for all $A \in
    F_{\Sigma}$, \[a(A) + b(A) \le \frac{|\mname{subformulas}(A)| +
      1}{2},\] where $|S|$ denotes the size of a finite set $S$.

  \textcolor{blue}{\textbf{Yunbing Weng, wengy12, 2/10/2019}}


      

  \item \textbf{[6 points]} Let $R_{\sf sf} \in F_{\Sigma} \times
    F_{\Sigma}$ be the relation such that \[A \mathrel{R_{\sf sf}} B\]
    iff $A$ is a subformula of $B$.  Prove that $(F_{\Sigma},R_{\sf
      sf})$ is a weak partial order but not a weak total order.

  \textcolor{blue}{\textbf{Yunbing Weng, wengy12, 2/10/2019}}

  \begin{proof}
  
  To check this we have to check it for 3 property of week partial order and proof that totality is false
  
  \begin{enumerate}
  
  \item  \textbf{Reflexive: } $\forall A \in F_{\Sigma}.$ A $R_{\sf sf}$ A

  For any formula, formula itself is a subformula, so this property holds

  \item \textbf{Antisymmetric: } $\forall A, B \in F_{\Sigma}.$(A $R_{\sf sf}$ B $\land$ B $R_{\sf sf}$ A $\implies$ A=B)

For any formula, subformula is an subset of symbols of formula, and for subset we have proved that $A \subseteq B \land B \subseteq A \iff A=B$

$\implies$ property holds

  \item \textbf{Transitive: } $\forall A, B, C \in F_{\Sigma}.$(A $R_{\sf sf}$ B $\land$ B $R_{\sf sf}$ C $\implies$ A $R_{\sf sf}$ C)

It is not hard to guess that we will use property of subformulas again and I think that it is pretty obvious that subset of a subset is still a subset

  \item \textbf{$\lnot$ Total: } $\exists A, B \in F_{\Sigma}$. $\lnot(A R_{\sf sf} B) \land \lnot(B R_{\sf sf} A)$

Let A = $(0 = 1)$ and B = $(0 < 1)$, this is good enough example I think

  \end{enumerate}
  
  
  
  \end{proof}

  \item \textbf{[4 points]} Let $A$ be any member of $F_{\Sigma}$ and
    $G = \mname{subformulas}(A)$.

  \be

    \item What are maximal elements of $G$ in $(F_{\Sigma},R_{\sf
        sf})$?

    \item What are minimal elements of $G$ in $(F_{\Sigma},R_{\sf
        sf})$?

    \item Does $G$ have a maximum element in $(F_{\Sigma},R_{\sf sf})$?  If
        so, what is it?

    \item Does $G$ have a minimum element in $(F_{\Sigma},R_{\sf
        sf})$?  If so, what is it?

  \ee

  \textcolor{blue}{\textbf{Yunbing Weng, wengy12, 2/10/2019}}

\begin{enumerate}

\item Maximal: A is an $R_{sf}-maximal$ element of $G \subseteq F_\Sigma$ if $A \in G$ and $\forall B \in F_\Sigma. B \in G \implies \lnot ($A$ R_{sf} $B$ )$

\item Minimal: A is an $R_{sf}-maximal$ element of $G \subseteq F_\Sigma$ if $A \in G$ and $\forall B \in F_\Sigma. B \in G \implies \lnot ($B$ R_{sf} $A$ )$

\item Maximum: A is the maximum element of G as all substrings are eather equal A or less then A.

\item Minimum: G does not have a minimum element as for example A = $(0 = 1) \lor (0 < 1)$ can be breaked down to itself, or $(0 = 1)$ or $(0 < 1)$, last two elements are unbreakeble and also neather of them is subset of other one. 

\end{enumerate}

\ee

\end{document}


