\documentclass[12pt]{article}

\usepackage{graphicx}
\usepackage{paralist}
\usepackage{amsfonts}
\usepackage{amsmath}
\usepackage{hhline}
\usepackage{booktabs}
\usepackage{multirow}
\usepackage{multicol}
\usepackage{url}

\oddsidemargin -10mm
\evensidemargin -10mm
\textwidth 160mm
\textheight 200mm
\renewcommand\baselinestretch{1.0}

\pagestyle {plain}
\pagenumbering{arabic}

\newcounter{stepnum}

%% Comments

\usepackage{color}

\newif\ifcomments\commentstrue

\ifcomments
\newcommand{\authornote}[3]{\textcolor{#1}{[#3 ---#2]}}
\newcommand{\todo}[1]{\textcolor{red}{[TODO: #1]}}
\else
\newcommand{\authornote}[3]{}
\newcommand{\todo}[1]{}
\fi

\newcommand{\wss}[1]{\authornote{blue}{SS}{#1}}

\title{Assignment 4, Part 1, Specification}
\author{SFWR ENG 2ME4}

\begin {document}

\maketitle
This Module Interface Specification (MIS) document contains modules, types and
methods for implementing the state of a game of Life.

\newpage

\section* {GameBoard Module}

\subsection*{Module}

GameBoard

\subsection* {Uses}

N/A

\subsection* {Syntax}

\subsubsection* {Exported Constants}

N/A

\subsubsection* {Exported Types}

GameBoard

\subsubsection* {Exported Access Programs}

\begin{tabular}{| l | l | l | p{5cm} |}
\hline
\textbf{Routine name} & \textbf{In} & \textbf{Out} & \textbf{Exceptions}\\
\hline
new GameBoard & string & & badbit, failbit, invalid\_argument\\
\hline
get &  & string & \\
\hline
size\_c & & $\mathbb{N}$ & \\
\hline
size\_r & & $\mathbb{N}$ & \\
\hline
update& &  & \\
\hline
clear& &  & \\
\hline
\end{tabular}

\subsection* {Semantics}

\subsubsection* {State Variables}

raw: seq of bool
board: seq of raw

\subsubsection* {State Invariant}

None

\subsubsection* {Assumptions \& Design Decisions}

\begin{itemize}
\item The GameBoard constructor is called before any
  other access routine is called for that object.  The constructor can only be
  called once. The constructor reads configuration from file. File should have format

***\#\#**

*******

\#\#*****

Where \# - cell with life, * - without each raw should have the same amount of cells
\end{itemize}

\subsubsection* {Access Routine Semantics}

new GameBoard($f$):
\begin{itemize}
\item transition: $board := while (line != EOF), raw.clear() (\forall i \in line.size(): line[i] == \# ? raw.add(true):raw.add(false)), board.add(raw)$
\item output: none
\item exception: $exc := (file.open(f) exeptions) || (line[i] != \# || line[i] != * || \exists i, j \in board.size(): board[i].size() != board[j].size()) \implies invalid\_argument$
\end{itemize}

\noindent get():
\begin{itemize}
\item output: $out := board converted to printable string$
\item exception: none
\end{itemize}

\noindent size\_c():
\begin{itemize}
\item output:  $out := board.size()$

\end{itemize}

\noindent size\_r():
\begin{itemize}
\item output: $out := board[0].size()$

\end{itemize}

\noindent update():
\begin{itemize}
\item transition: $board := updated\_buffer()$
\end{itemize}

\noindent clear():
\begin{itemize}
\item transition: $board := null$
\end{itemize}


\subsection*{Local Variables}

\noindent $\text{temp\_board} : \text{seq of seq of bool}$

\subsection*{Local Functions}

\noindent $\text{updated\_buffer} : \text{void} \rightarrow \text{seq of seq of bool}$\\
\noindent $\text{updated\_buffer()} \equiv \forall i \in board.size() : \forall j \in board[i].size() : temp\_board = update\_sell(i ,j)$

\noindent $\text{update\_sell} : \mathbb{N} \times \mathbb{N} \rightarrow \text{bool}$\\
\noindent $\text{updated\_sell(i, j)} \equiv $

\begin{tabular}{|p{4cm}|p{7cm}|l|}
\hhline{|-|-|-|}
   board[i][j] == true & neighbors\_number(i,j) $<$ 2 & false\\
\hhline{|~|-|-|}
 & neighbors\_number(i,j) $>$ 3 & false\\
\hhline{|~|-|-|}
 & neighbors\_number(i,j) $>$ 1 and neighbors\_number(i,j) $<$ 4 & false\\
\hhline{|-|-|-|}
  $board[i][j] == false$ & neighbors\_number(i,j) $==$ 3 & true\\
\hhline{|~|-|-|}
 & neighbors\_number(i,j) $!=$ 3 & false\\
\hhline{|-|-|-|}
\end{tabular}\\
\begin{itemize}
\item exception: i $<$ 0 $\|$ i $>$ board.size() $\|$ j $<$ 0 $\|$ j $>$ board[i].size() $\rightarrow$ out\_of\_range
\end{itemize}

\noindent $\text{neighbors\_number} : \mathbb{N} \times \mathbb{N} \rightarrow \mathbb{N}$\\
\noindent $\text{neighbors\_number(i, j)} \equiv (+\forall c \in i \pm 1 : \forall d \in j \pm 1: (c != i \&\& d != j \&\& c < board.size() \&\& c > 0 \&\& d < board[c].size() \&\& d > 0): if(board[c][d]): 1  $)
\begin{itemize}
\item exception: i $<$ 0 $\|$ i $>$ board.size() $\|$ j $<$ 0 $\|$ j $>$ board[i].size() $\rightarrow$ out\_of\_range
\end{itemize}

\newpage

\section* {Gui Module}

\subsection*{Module}

Gui

\subsection* {Uses}

Gameboard

\subsection* {Syntax}

\subsubsection* {Exported Constants}

N/A

\subsubsection* {Exported Types}

N/A

\subsubsection* {Exported Access Programs}

\begin{tabular}{| l | l | l | p{5cm} |}
\hline
\textbf{Routine name} & \textbf{In} & \textbf{Out} & \textbf{Exceptions}\\
\hline
action & & & \\
\hline
\end{tabular}

\subsection* {Semantics}

\subsubsection* {State Variables}

key: current input string
board: GameBoard
cmd: terminal

\subsubsection* {State Invariant}

None

\subsubsection* {Assumptions \& Design Decisions}

\begin{itemize}
\item This is the main control and view Module, the main function "action" that always will
take input from console and decide what to do depends on what user input is:
"u" - update and refresh the showing board

"o filename" - output the current state to the file

"start filename" - restart the game with the initial configuration specified in file

"q" - quit without saving

"HELP" - print help massage

Initially it'll show the welcome massage
\end{itemize}

\subsubsection* {Access Routine Semantics}

action():
\begin{itemize}
\item transition: on first run: print\_help\_massage()

if(key == "start filename"): board.clear(), board = new Gameboard(filename), cmd.clear(), print\_current\_board()

else if(key == "u"): board.update(), cmd.clear(), print\_current\_board()

else if(key == "o filename"): output\_to\_file(filename)

else if(key == "q"): board.clear(), cmd.clear(), return

else if(key == "HELP"): print\_help\_massage()

else: print\_error\_massage()

\end{itemize}

\subsection*{Local Variables}

\subsection*{Local Functions}

\noindent $\text{cmd.clear} : \text{void} \rightarrow \text{void}$\\
\noindent $\text{cmd.clear()} \equiv $ function from cstdlib: system("CLS")\\


\noindent $\text{print\_help\_massage} : \text{void} \rightarrow \text{void}$\\
\noindent $\text{print\_help\_massage()} \equiv $ prints welcome massage:

please type what you want and press enter:

"u" - update and refresh the showing board

"o filename" - output the current state to the file

"start filename" - restart the game with the initial configuration specified in file

"q" - quit without saving

"HELP" - print help massage\\


\noindent $\text{print\_current\_board} : \text{void} \rightarrow \text{void}$\\
\noindent $\text{print\_current\_board()} \equiv $ prints: "current board" + endl + board.get() + endl\\


\noindent $\text{output\_to\_file} : \text{string} \rightarrow \text{void}$\\
\noindent $\text{output\_to\_file(f)} \equiv $ file = File.open(f); fstream board.get() to it
\begin{itemize}
\item exception: $exc := (file.open(f) exeptions)$
\end{itemize}

\noindent $\text{print\_error\_massage} : \text{void} \rightarrow \text{void}$\\
\noindent $\text{print\_error\_massage()} \equiv $ print: "try again, you can do it! If you have any questions, you can massage me wengyunbing@gmail.com"





\end {document}
